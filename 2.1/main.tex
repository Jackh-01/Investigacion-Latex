\documentclass{article}
\usepackage[utf8]{inputenc}
\usepackage{fancyhdr, graphicx, parskip}
\usepackage[spanish]{babel}
\usepackage{listings}
\usepackage{graphicx}
\graphicspath{ {images} }
\usepackage{cite}
\usepackage [colorlinks = true, 
            linkcolor = blue,
            citecolor = black,
            urlcolor = blue]{hyperref}
            

\begin{document}
\begin{titlepage}
\renewcommand{\headrulewidth}{3pt}
\fancyhead[L]{}
\fancyhead[R]{}
    \includegraphics[width=4cm]{imagen.png}
    \begin{center}
        \vspace*{1cm} 
            
        \Huge
        \textbf{Notacion de la memoria del computador}
            
        \vspace{0.8 cm
}
        
        \LARGE
        Taller de la memoria 
            
        \vspace{1.5cm}
            
        \textbf{Jackh Emmanuel Narvaez Guerra}
            
        \vfill
            
        \vspace{0.8cm}
            
        \Large
        Despartamento de Ingeniería Electrónica y Telecomunicaciones\\
        Universidad de Antioquia\\
        Medellín\\
        Septiembre de 2020
            
    \end{center}
\end{titlepage}
\thispagestyle{fancy}

\tableofcontents

\section{Introducción}

\section{Defina que es la memoria del computador.}
Esta es la primera sección, podemos agregar algunos elementos adicionales y todo será escrito correctamente. Más aún, si una palabra es demasiado larga y tiene que ser truncada, babel tratará de truncarla correctamente dependiendo del idioma.

\section{Mencione los tipos de memoria que conozco.} \label{contenido}

Esta sección es para ver qué pasa con los comandos 
que definen texto

El paquete también agrega un comportamiento especial 
a <<estas marcas para hacer citas textuales>> tal como 
lo indican las reglas de la RAE. \cite{dirac}


\section{Describa la manera como se gestiona la memoria en un computador.}

en esta seccion se hablara respecto a la memoria del computador 


\section{¿Qué hace que una memoria sea más rápida que otra?}
\begin{lstlisting}
#include <stdio.h>
#define N 10
/* Block
 * comment */

int main()
{
    int i;

    // Line comment.
    puts("Hello world!");
    
    for (i = 0; i < N; i++)
    {
        puts("LaTeX is also great for programmers!");
    }

    return 0;
}
\end{lstlisting}

A continuación se presenta el logo de C++ Figura (\ref{fig:cpplogo})

\begin{figure}[h]
\includegraphics[width=4cm]{cpplogo.png}
\centering
\caption{Logo de C++}
\label{fig:cpplogo}
\end{figure}

En la sección de teoremas (\ref{contenido})

\section{Conclusión} \label{conclulsion}

\bibliographystyle{IEEEtran}
\bibliography{references}

\end{document}
